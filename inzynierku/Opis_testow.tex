\chapter{Testy}
\section{Test podążania po linii}
\subsection{Cel testu}
Test ma na celu sprawdzenie poprawności implementacji czujników odbiciowych wraz z wykazaniem poprawności integracji silników z odczytami.
\subsection{Opis testu}
Po uruchomieniu robota przeprowadzono kalibrację. Następnie ustawiono konstrukcję na trasie wyznaczonej przez czarną linię. Owa trasa zawierała zakręty pod kątem prostym odpowiednio w lewo, dwukrotnie w prawo oraz w lewo. Test przeprowadzono dziesięć razy.
\subsection{Rezultaty testu}
Podczas testu robot raz zjechał z wyznaczonej trasy pomijając drugi zakręt w prawo. Wszystkie pozostałe przejazdy zakończyły się sukcesem poprzez osiągnięcie końca trasy.

\section{Test wykrywania przeszkody}
Test ma na celu sprawdzenie poprawności implementacji czujników ultradźwiękowych wraz z wykazaniem poprawności integracji silników z odczytami.
\subsection{Cel testu}
\subsection{Opis testu}
\subsection{Rezultaty testu}

\section{Test podążania wzdłuż przeszkody}
Test ma na celu sprawdzenie zachowania podczas utrzymywania stałej odległości od przeszkody, tym samym sprawdzenie poprawności omijania przeszkody.
\subsection{Cel testu}
\subsection{Opis testu}
\subsection{Rezultaty testu}

\section{Test powrotu na linię}
Test ma na celu sprawdzenie zachowania po ominięciu przeszkody i wykazania, czy robot jest w stanie powrócić na tor jazdy.
\subsection{Cel testu}
\subsection{Opis testu}
\subsection{Rezultaty testu}