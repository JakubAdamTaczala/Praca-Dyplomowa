\chapter{Podsumowanie}
\section{Zadania wykonane w ramach pracy}
W ramach pracy dyplomowej został zaprojektowany i wykonany robot w układzie (2, 0). W celu osiągnięcia optymalnych parametrów robota przeanalizowano konstrukcje, korzystające z podobnych metod określenia położenia w otoczeniu. Na tej podstawie dobrano czujnik wykrywania linii, czujniki ultradźwiękowe oraz sposób napędzania platformy z zastosowaniem silników krokowych. Po zapoznaniu się z dobranymi czujnikami zdecydowano o płytce rozruchowej kierując się wymaganiami specyfikacji określonymi przez czujniki.

Analizując sposoby poruszania się wspomnianych robotów zaprojektowano algorytm dla tworzonego projektu. Uwzględniono ograniczenia, jakie wynikają z zastosowanych czujników, Następnie opracowano obsługę czujników i algorytm sterowania. 

Gotową konstrukcje przetestowano sprawdzając poprawność implementacji, poddając ją próbom w rzeczywistym otoczeniu. Podczas testów została określona poprawność toru jazdy przy pomocy czujnika odbiciowego. Przeprowadzono również pomiary odległości od przeszkody czujnikami ultradźwiękowymi. Wykorzystując owe pomiary określono sposób poruszania się przy przeszkodzie. Natomiast zliczanie kroków pozwoliło na określenie przybliżonego dystansu pokonanego od startu robota, wraz z określeniem odległości poza wyznaczoną trasą. Korzystając z zastosowanych modułów wykonano próby odnalezienia linii wyznaczającej dalszy tor jazdy, podczas poruszania się wzdłuż omijanego przedmiotu.

\section{Uzyskane rezultaty}
UZUPEŁNIĆ!!!!!!!!!!!!!!!!

\section{Możliwe rozszerzenia}
Możliwości przedstawionej konstrukcji mogą być rozszerzone przez dodanie kamery. Pozwoli to na ograniczenie ilości czujników potrzebnych do wykrywania przeszkody oraz określenia położenia względem niej. Dodatkowo można wykorzystać ją do śledzenia linii ograniczając tym ilości czujników. Takie rozwiązanie wymaga większej mocy obliczeniowej do celu przetwarzania uzyskanego obrazu.

Kolejnym rozszerzeniem może być użycie większej ilości zastosowanych czujników uzyskując tym samym podgląd na otoczenie w każdym kierunku.