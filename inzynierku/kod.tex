#include <QTRSensors.h>

#define LED           13    // dioda sygnalizujaca
#define NUM_SENSORS   8     // ilosc sensorow lini
#define TIMEOUT       2500  // czas w 2500 mikrosekundach dlaustawienia wyjscia QRT na niskie for sensor outputs to go low
#define EMITTER_PIN   2     // Emiter QRT
#define TRIG_L        49    // Czujniki ultradzwiekowe
#define ECHO_L        48
#define TRIG_S        51
#define ECHO_S        50
#define TRIG_P        53
#define ECHO_P        52
#define STEP_P        37    // Silniki krokowe
#define DIR_P         36
#define STEP_L        39
#define DIR_L         38
#define DEGRE         1.8

//####################__IMPLEMENTACJA_CZUJNIKA_LINII
QTRSensorsRC qtrrc((unsigned char[]) {40, 41, 42, 43, 44, 45, 46, 47}, NUM_SENSORS, TIMEOUT, EMITTER_PIN); 
unsigned int sensorValues[NUM_SENSORS]; // tablica odczytow QRT

void miganie_LED(int PIN, int d_time) {
  digitalWrite(PIN, HIGH);
  delay(d_time);
  digitalWrite(PIN, LOW);
  delay(d_time);
}

void QRT_kalibracja(){
  delay(500);
  pinMode(LED, OUTPUT);
  digitalWrite(13, HIGH);        // informacja o ty, ze trwa kalibracja
  for (int i = 0; i < 400; i++)  // kalibracja trwa okolo 10s
    qtrrc.calibrate();           // odczytuje wszystkie sensory 10 razy przez 2500 us na odczyt (tj. ~25 ms na polacznie)
  digitalWrite(13, LOW);         //koniec kalibracji
}

void QRT_kalibracja_wartosci_MIN(){ // Wypisanie minimalnych wartosci kalibracji
   for (int i = 0; i < NUM_SENSORS; i++)
  {
    Serial.print(qtrrc.calibratedMinimumOn[i]);
    Serial.print(' ');
  }
  Serial.println();
}

void QRT_kalibracja_wartosci_MAX(){ // Wypisanie maksymalnych wartosci kalibracji
  for (int i = 0; i < NUM_SENSORS; i++)
  {
    Serial.print(qtrrc.calibratedMaximumOn[i]);
    Serial.print(' ');
  }
  Serial.println();
}

void QRT_START(){
  QRT_kalibracja();
  QRT_kalibracja_wartosci_MIN();
  QRT_kalibracja_wartosci_MAX();
  Serial.println();
  delay(1000);
}

void QRT_odczyt(){
  qtrrc.read(sensorValues);
}

void QRT_wypisanie_wartosci(){
  for (unsigned char i = 0; i < NUM_SENSORS; i++)
  {
    Serial.print(i+1);
    Serial.print('\t');
    Serial.print(sensorValues[i]);
    Serial.print('\t');
  }
  Serial.println();
}

int QRT_ktory_widzi(){
  int w=0;
  int ktory=0;
  
  for (unsigned char i = 0; i < NUM_SENSORS; i++)
  {
    if(w<sensorValues[i])
    {
      w=sensorValues[i];
      ktory=i+1;
    }
    if(w<1500)
    {
      ktory=0;
    }
  }
  return ktory;
}

void US_START(){
  pinMode(TRIG_L, OUTPUT); // Pin TRIG jako wyjście
  pinMode(ECHO_L, INPUT);  // Pin echo jako wejście
  pinMode(TRIG_S, OUTPUT); // Pin TRIG jako wyjście
  pinMode(ECHO_S, INPUT);  // Pin echo jako wejście
  pinMode(TRIG_P, OUTPUT); // Pin TRIG jako wyjście
  pinMode(ECHO_P, INPUT);  // Pin echo jako wejście
}

void US_POMIAR(int PIN_TRIG){
  digitalWrite(PIN_TRIG, LOW);
  delayMicroseconds(2);
  digitalWrite(PIN_TRIG, HIGH);
  delayMicroseconds(10);
  digitalWrite(PIN_TRIG, LOW);
}

int US_ODCZYT_cm(int PIN_ECHO){
  long US_czas, US_odleglosc;
  US_czas = pulseIn(PIN_ECHO, HIGH);
  US_odleglosc = US_czas/58;
  return US_odleglosc;
}

void stepperRotate_L(float rotation, float rpm) {
  float stepsPerRotation = (360.00/DEGRE);
  unsigned long stepPeriodmicroSec = ((60.0000 / (rpm * stepsPerRotation)) * 1E6 / 2.0000) - 5;
  for (unsigned long i = 0; i < rotation; i++) {
    digitalWrite(STEP_L,HIGH); // kierunek
    delayMicroseconds(stepPeriodmicroSec);
    digitalWrite(STEP_L,LOW); // kierunek
    delayMicroseconds(stepPeriodmicroSec);
  }
}

void stepperRotate_P(float rotation, float rpm) {
  float stepsPerRotation = (360.00/DEGRE);
  unsigned long stepPeriodmicroSec = ((60.0000 / (rpm * stepsPerRotation)) * 1E6 / 2.0000) - 5;
  for (unsigned long i = 0; i < rotation; i++) {
    digitalWrite(STEP_P,HIGH); // kierunek
    delayMicroseconds(stepPeriodmicroSec);
    digitalWrite(STEP_P,LOW); // kierunek
    delayMicroseconds(stepPeriodmicroSec);
  }
}

void stepperRotate_S(float rotation, float rpm) {
  float stepsPerRotation = (360.00/DEGRE);
  unsigned long stepPeriodmicroSec = ((60.0000 / (rpm * stepsPerRotation)) * 1E6 / 2.0000) - 5;
  for (unsigned long i = 0; i < rotation; i++) {
    digitalWrite(STEP_L,HIGH); // kierunek
    digitalWrite(STEP_P,HIGH); // kierunek
    delayMicroseconds(stepPeriodmicroSec);
    digitalWrite(STEP_L,LOW); // kierunek
    digitalWrite(STEP_P,LOW); // kierunek
    delayMicroseconds(stepPeriodmicroSec);
  }
}

void setup() {
  pinMode(LED, OUTPUT); // Ustawienie portu led jako wyjscie
  Serial.begin(9600);   // Ustawienie prędkości transmisji
  //QRT_START();          // Kalibracja i sprawdzenie QRT
  //US_START();           // Ustawienie wejsc/wyjsc

  pinMode(STEP_P, OUTPUT); // Ustawienie portow na wyjscia dla silnikow
  pinMode(DIR_P, OUTPUT);
  pinMode(STEP_L, OUTPUT);
  pinMode(DIR_L, OUTPUT);
  
  digitalWrite(DIR_P,HIGH); // kierunek do przodu
  digitalWrite(DIR_L,HIGH); // kierunek do przodu
}

void loop() {
  long US_odleglosc_L;
  long US_odleglosc_S;
  long US_odleglosc_P;
  
  //miganie_LED(LED, 500);

  /*
  QRT_odczyt();
  //QRT_wypisanie_wartosci();
  Serial.println(QRT_ktory_widzi());

  US_POMIAR(TRIG_L);                    // Wykonanie pomiaru
  US_odleglosc_L=pulseIn(ECHO_L, HIGH); // Pobranie wyniku pomiaru
  US_POMIAR(TRIG_S);
  US_odleglosc_S=pulseIn(ECHO_S, HIGH);
  US_POMIAR(TRIG_P);
  US_odleglosc_P=pulseIn(ECHO_P, HIGH);
  
  Serial.print(US_odleglosc_L);         // Wypisanie pomiarow
  Serial.print(" ");
  Serial.print(US_odleglosc_S);
  Serial.print(" ");
  Serial.print(US_odleglosc_P);
  Serial.println(" ");
  */

  stepperRotate_S(10, 50);
  stepperRotate_P(20, 50);
  stepperRotate_L(20, 50);
}